\documentclass[11pt,preprint, authoryear]{elsarticle}

\usepackage{lmodern}
%%%% My spacing
\usepackage{setspace}
\setstretch{1.2}
\DeclareMathSizes{12}{14}{10}{10}

% Wrap around which gives all figures included the [H] command, or places it "here". This can be tedious to code in Rmarkdown.
\usepackage{float}
\let\origfigure\figure
\let\endorigfigure\endfigure
\renewenvironment{figure}[1][2] {
    \expandafter\origfigure\expandafter[H]
} {
    \endorigfigure
}

\let\origtable\table
\let\endorigtable\endtable
\renewenvironment{table}[1][2] {
    \expandafter\origtable\expandafter[H]
} {
    \endorigtable
}


\usepackage{ifxetex,ifluatex}
\usepackage{fixltx2e} % provides \textsubscript
\ifnum 0\ifxetex 1\fi\ifluatex 1\fi=0 % if pdftex
  \usepackage[T1]{fontenc}
  \usepackage[utf8]{inputenc}
\else % if luatex or xelatex
  \ifxetex
    \usepackage{mathspec}
    \usepackage{xltxtra,xunicode}
  \else
    \usepackage{fontspec}
  \fi
  \defaultfontfeatures{Mapping=tex-text,Scale=MatchLowercase}
  \newcommand{\euro}{€}
\fi

\usepackage{amssymb, amsmath, amsthm, amsfonts}

\def\bibsection{\section*{References}} %%% Make "References" appear before bibliography


\usepackage[round]{natbib}

\usepackage{longtable}
\usepackage[margin=2.3cm,bottom=2cm,top=2.5cm, includefoot]{geometry}
\usepackage{fancyhdr}
\usepackage[bottom, hang, flushmargin]{footmisc}
\usepackage{graphicx}
\numberwithin{equation}{section}
\numberwithin{figure}{section}
\numberwithin{table}{section}
\setlength{\parindent}{0cm}
\setlength{\parskip}{1.3ex plus 0.5ex minus 0.3ex}
\usepackage{textcomp}
\renewcommand{\headrulewidth}{0.2pt}
\renewcommand{\footrulewidth}{0.3pt}

\usepackage{array}
\newcolumntype{x}[1]{>{\centering\arraybackslash\hspace{0pt}}p{#1}}

%%%%  Remove the "preprint submitted to" part. Don't worry about this either, it just looks better without it:
\makeatletter
\def\ps@pprintTitle{%
  \let\@oddhead\@empty
  \let\@evenhead\@empty
  \let\@oddfoot\@empty
  \let\@evenfoot\@oddfoot
}
\makeatother

 \def\tightlist{} % This allows for subbullets!

\usepackage{hyperref}
\hypersetup{breaklinks=true,
            bookmarks=true,
            colorlinks=true,
            citecolor=blue,
            urlcolor=blue,
            linkcolor=blue,
            pdfborder={0 0 0}}


% The following packages allow huxtable to work:
\usepackage{siunitx}
\usepackage{multirow}
\usepackage{hhline}
\usepackage{calc}
\usepackage{tabularx}
\usepackage{booktabs}
\usepackage{caption}


\newenvironment{columns}[1][]{}{}

\newenvironment{column}[1]{\begin{minipage}{#1}\ignorespaces}{%
\end{minipage}
\ifhmode\unskip\fi
\aftergroup\useignorespacesandallpars}

\def\useignorespacesandallpars#1\ignorespaces\fi{%
#1\fi\ignorespacesandallpars}

\makeatletter
\def\ignorespacesandallpars{%
  \@ifnextchar\par
    {\expandafter\ignorespacesandallpars\@gobble}%
    {}%
}
\makeatother

\newlength{\cslhangindent}
\setlength{\cslhangindent}{1.5em}
\newenvironment{CSLReferences}%
  {\setlength{\parindent}{0pt}%
  \everypar{\setlength{\hangindent}{\cslhangindent}}\ignorespaces}%
  {\par}


\urlstyle{same}  % don't use monospace font for urls
\setlength{\parindent}{0pt}
\setlength{\parskip}{6pt plus 2pt minus 1pt}
\setlength{\emergencystretch}{3em}  % prevent overfull lines
\setcounter{secnumdepth}{5}

%%% Use protect on footnotes to avoid problems with footnotes in titles
\let\rmarkdownfootnote\footnote%
\def\footnote{\protect\rmarkdownfootnote}
\IfFileExists{upquote.sty}{\usepackage{upquote}}{}

%%% Include extra packages specified by user

%%% Hard setting column skips for reports - this ensures greater consistency and control over the length settings in the document.
%% page layout
%% paragraphs
\setlength{\baselineskip}{12pt plus 0pt minus 0pt}
\setlength{\parskip}{12pt plus 0pt minus 0pt}
\setlength{\parindent}{0pt plus 0pt minus 0pt}
%% floats
\setlength{\floatsep}{12pt plus 0 pt minus 0pt}
\setlength{\textfloatsep}{20pt plus 0pt minus 0pt}
\setlength{\intextsep}{14pt plus 0pt minus 0pt}
\setlength{\dbltextfloatsep}{20pt plus 0pt minus 0pt}
\setlength{\dblfloatsep}{14pt plus 0pt minus 0pt}
%% maths
\setlength{\abovedisplayskip}{12pt plus 0pt minus 0pt}
\setlength{\belowdisplayskip}{12pt plus 0pt minus 0pt}
%% lists
\setlength{\topsep}{10pt plus 0pt minus 0pt}
\setlength{\partopsep}{3pt plus 0pt minus 0pt}
\setlength{\itemsep}{5pt plus 0pt minus 0pt}
\setlength{\labelsep}{8mm plus 0mm minus 0mm}
\setlength{\parsep}{\the\parskip}
\setlength{\listparindent}{\the\parindent}
%% verbatim
\setlength{\fboxsep}{5pt plus 0pt minus 0pt}



\begin{document}



%titlepage
\thispagestyle{empty}
\begin{center}
\begin{minipage}{0.75\linewidth}
    \centering
%Entry1
    {\uppercase{\huge Real Exchange Rate Behaviour: A Replication and
Robustness Check\par}}
    \vspace{2cm}
%Author's name
    {\LARGE Cassandra Pengelly \textbar{} 20346212\par}
    \vspace{1cm}
%University logo
\begin{center}
    \includegraphics[width=0.5\linewidth]{Tex/Logo.png}
\end{center}
\vspace{1cm}
%Supervisor's Details
\begin{center}
    {\Large Econometrics 871: Time Series Project\par}
    \vspace{1cm}
%Degree
    {\large \par}
    \vspace{1cm}
%Institution
    {\large \par}
    \vspace{1cm}
%Date
    {\large }
%More
    {\normalsize }
%More
    {\normalsize }
\end{center}
\end{minipage}
\end{center}
\clearpage


\begin{frontmatter}  %

\title{}

% Set to FALSE if wanting to remove title (for submission)


\vspace{1cm}





\vspace{0.5cm}

\end{frontmatter}


\renewcommand{\contentsname}{Table of Contents}
{\tableofcontents}

%________________________
% Header and Footers
%%%%%%%%%%%%%%%%%%%%%%%%%%%%%%%%%
\pagestyle{fancy}
\chead{}
\rhead{}
\lfoot{}
\rfoot{\footnotesize Page \thepage}
\lhead{}
%\rfoot{\footnotesize Page \thepage } % "e.g. Page 2"
\cfoot{}

%\setlength\headheight{30pt}
%%%%%%%%%%%%%%%%%%%%%%%%%%%%%%%%%
%________________________

\headsep 35pt % So that header does not go over title




\newpage

\hypertarget{introduction}{%
\section{\texorpdfstring{Introduction
\label{Introduction}}{Introduction }}\label{introduction}}

How do we compare living standards and economic productivity between
countries? This is one of the questions that macroeconomics attempts to
answer, and a number of tools have been developed within the field to
this end. One of these tools is the Purchasing Power Parity (PPP)
theory, which uses a basket of goods to compare the currencies of
different countries. This theory has been widely tested using data, and
the results have been divisive and somewhat puzzling
(\protect\hyperlink{ref-puz}{El-Gamal \& Ryu}
(\protect\hyperlink{ref-puz}{2006})). In this essay, I
replicate\footnote{More accurately, try my best to replicate} the paper
``Real Exchange Rate Behaviour: Evidence from Black Markets'' by
\protect\hyperlink{ref-Kul}{Luintel}
(\protect\hyperlink{ref-Kul}{2000}), which tests the PPP hypothesis.
\protect\hyperlink{ref-Kul}{Luintel} (\protect\hyperlink{ref-Kul}{2000})
finds that the behaviour of the real exchange rate is mean-reverting in
the long-run, which suggests that the PPP theory is empirically
supported. I include some other tests in addition to those presented in
the paper as a robustness check on these results.

This essay\footnote{This essay was written in R using the package by
  \protect\hyperlink{ref-Texevier}{Katzke}
  (\protect\hyperlink{ref-Texevier}{2017})} is organised as follows.
Section \ref{Context} contextualises Luintel's paper and discusses the
robustness checks. Section \ref{Data} discusses the data and reports the
results of the Wald-Wolfowitz tests. Section \ref{Unit} deals with the
unit root tests and section \ref{Var} reports the results of the
variance ratio test. The code for this replication can be found on
Github \href{https://github.com/cass-code/metrics.git}{here}.

\hypertarget{context-and-evaluation}{%
\section{\texorpdfstring{Context and Evaluation
\label{Context}}{Context and Evaluation }}\label{context-and-evaluation}}

\protect\hyperlink{ref-Kul}{Luintel} (\protect\hyperlink{ref-Kul}{2000})
investigates whether the PPP hypothesis holds empirically. To test this
theory, \protect\hyperlink{ref-Kul}{Luintel}
(\protect\hyperlink{ref-Kul}{2000}) uses monthly black market real
exchange rates (in terms of the US dollar) from eight developing Asian
countries: India, Sri Lanka, Myanmar, Malaysia, Pakistan, Philippines,
Taiwan and Thailand. Using data from developing countries (rather than
from developed countries) was a novel approach for its time. The black
market rates are used as a proxy for the float rates of developing
countries.

Practically, the paper has two main aims: the first is to determine
whether there are segmented trends in the data, and the second is to
test whether the panel data is stationary. At the time that this paper
was written (early 2000s), the puzzle of PPP was that tests for unit
roots failed to reject the null hypothesis. The null hypothesis in these
cases was the presence of unit roots; these tests implied
non-stationarity and discredited PPP, despite the support from economic
theory(\protect\hyperlink{ref-puz}{El-Gamal \& Ryu}
(\protect\hyperlink{ref-puz}{2006})).

\protect\hyperlink{ref-Kul}{Luintel} (\protect\hyperlink{ref-Kul}{2000})
makes use of (more) powerful unit root tests for heterogeneous panels,
and finds that real exchange rates are mean-reverting. This was novel
for the time as most time-series studies rejected PPP and concluded that
the real exchange rate followed a random walk. This suggested that any
shocks to the real exchange rate were persistent and there was no
mean-reversion either in the short or long term
(\protect\hyperlink{ref-rog}{Rogoff}
(\protect\hyperlink{ref-rog}{1996})).
\protect\hyperlink{ref-Kul}{Luintel} (\protect\hyperlink{ref-Kul}{2000})
finds that the black market real exchange rates do not behave in an
excessively volatile manner, which conflicted with the findings of the
literature at that time. Additionally, the findings of the study implied
that such empirical investigations may not necessarily suffer from
survivorship bias.

A critical part of Luintel's paper is testing for unit roots in the
panel data; specifically, the paper makes use of the Im-Pesaran-Shin
(IPS) T-bar test. In addition to replicating this test, I implement
several other unit root tests as a robustness check and find that the
results are mixed. \protect\hyperlink{ref-Kul}{Luintel}
(\protect\hyperlink{ref-Kul}{2000: 170}) defends the choice of the IPS
tests well, citing that they allow for the dynamics and error variances
across groups and these tests may have better small sample properties. I
run the IPS tests using \protect\hyperlink{ref-Kul}{Luintel}
(\protect\hyperlink{ref-Kul}{2000})'s specified lags, and the AIC
method. I then implement the panel stationarity tests proposed by
\protect\hyperlink{ref-lev}{Levin, Lin \& James Chu}
(\protect\hyperlink{ref-lev}{2002}), \protect\hyperlink{ref-wu}{Maddala
\& Wu} (\protect\hyperlink{ref-wu}{1999}),
\protect\hyperlink{ref-had}{Hadri} (\protect\hyperlink{ref-had}{2002}),
as well as a bootstrapped panel unit root test from
\protect\hyperlink{ref-pal}{Palm, Smeekes \& Urbain}
(\protect\hyperlink{ref-pal}{2011})..

\hypertarget{data}{%
\section{\texorpdfstring{Data \label{Data}}{Data }}\label{data}}

The data used for the analysis is a series on black market nominal
exchange rates and consumer price indices (CPI) for 8 developing Asian
countries, namely: India, Sri Lanka, Myanmar, Malaysia, Pakistan,
Philippines, Taiwan and Thailand. I take a subset of these countries by
excluding Taiwan\footnote{I excluded Taiwan because there is some data
  missing from the set and I don't know how to manage an unbalanced
  panel. However, it is also interesting to test if the results of the
  paper hold when taking a subset of the data.} from the analysis.
\protect\hyperlink{ref-Kul}{Luintel} (\protect\hyperlink{ref-Kul}{2000})
sources data from various issues of \emph{Pick's Currency Year Book} and
\emph{World Currency Year Book}. The data used for Luintel's paper is
accessible through the Journal of Applied Econometrics archive, which is
where I attained my data. The sample period runs for 31 periods from
January 1958 to June 1989. This sample period is split into two parts:
Bretton Woods and after Bretton Woods (also referred to as pre-float
period and the float period).

The nominal exchange rates are units currencies per unit of US dollar.
There were two mistakes in the nominal exchange rate datasets: for
Myanmar November 1974, there was a value of 1.45, which I replaced with
16.5 (based on interpolation). And for the Philippines in September
1975, there was a value of 0.7 with which I replaced with 7.7 (based on
interpolation).\footnote{I discovered these mistakes when there was a
  dramatic difference in my plots of the real exchange rates and
  Luintel's plots.} Luintel sources the CPI figures from various issues
of International Financial Statistics (which are included in Luintel's
dataset available in the JAE data archives).

To calculate the real exchange rates, I follow the lead of
\protect\hyperlink{ref-Kul}{Luintel} (\protect\hyperlink{ref-Kul}{2000:
165}) and apply the following formula to the nominal exchange rates:

\[
rex = log(Nominal Exchange Rate) - log(CPI) + log(United States CPI)
\]

I plot the real exchange rate series below in \ref{Figure1}. The plots
below match those of \protect\hyperlink{ref-Kul}{Luintel}
(\protect\hyperlink{ref-Kul}{2000: 166}) and preliminarily indicate that
the real exchange rates are trending. Additionally, the graphs show that
the black market exchange rates are somewhat volatile. As expected, we
see that after the first oil shock of 1973 the currencies appreciated
and then slowly reverted. The plots suggest that the trends are
segmented. \protect\hyperlink{ref-Kul}{Luintel}
(\protect\hyperlink{ref-Kul}{2000: 169}) tests this hypothesis using
formal tests, and I follow suit - the results of the Wald-Wolfowitz
Tests are reported below after the plots, in table \ref{wald}.

\begin{center}\includegraphics{20346212_files/figure-latex/Figure1-1} \end{center}

\begin{figure}[H]

{\centering \includegraphics{20346212_files/figure-latex/Figure2-1} 

}

\caption{Plot of Real Exchange Rates over Time\label{Figure1}}\label{fig:Figure2}
\end{figure}

\hypertarget{wald-wolfowitz-tests}{%
\subsection{\texorpdfstring{Wald-Wolfowitz Tests
\label{wald}}{Wald-Wolfowitz Tests }}\label{wald-wolfowitz-tests}}

The Wald-Wolfowitz test is a nonparametric test that discriminates
between the underlying distributions of the Bretton Woods and post
Bretton Woods real exchange rates. Essentially, it tests whether two
random samples are from populations with the same distribution (this is
the null hypothesis), or whether the two samples descend from
populations with different distributions (the alternative
hypothesis).\footnote{\protect\hyperlink{ref-Kul}{Luintel}
  (\protect\hyperlink{ref-Kul}{2000: 169}) gives the mathematical
  details of the test.}

The critical values for this test at 1\% and 5\% are 2.58 and 1.96
respectively. \ref{Wtable} shows that the tests reject the null
hypothesis at a 1\% significance level for all the countries. These
results imply that the Bretton Woods real exchange rates descend from a
population that follows a distribution that may differ in skewness,
kurtosis and dispersion from that of the post Bretton Woods. This
suggests that it is important to include the Bretton Woods period in our
analysis of real exchange rates. \protect\hyperlink{ref-Kul}{Luintel}
(\protect\hyperlink{ref-Kul}{2000: 169}) reports smaller test
statistics, but rejects the null comfortably for all of the countries.

\begin{table}[H]
\centering
\caption{Wald-Wolfowitz tests} 
\label{Wtable}
\begin{tabular}{lrrrrrrr}
  \hline
Test/Country & India & SriLanka & Malaysia & Myanmar & Pakistan & Philippines & Thailand \\ 
  \hline
Wald-Wolfowitz & -16.07 & -18.54 & -17.10 & -18.23 & -16.27 & -17.10 & -15.96 \\ 
   \hline
\end{tabular}
\end{table}

\hypertarget{unit-root-tests}{%
\section{\texorpdfstring{Unit Root Tests
\label{Unit}}{Unit Root Tests }}\label{unit-root-tests}}

We can define relative PPP as: \[
\Delta s_{t}=\Delta p_{t}-\Delta p_{t}^{*} \label{eq1}
\] This relationship shows that the percentage change in the nominal
exchange rate should be equal to the difference in inflation between the
domestic and foreign country. The real exchange rate
\(\left(q_{t}\right)\) is given by: \[
q_{t}=s_{t}-p_{t}+p_{t}^{*} \label{eq2}
\] From \ref{eq1} we can see that the real exchange rate should be zero
or a constant if PPP holds continuously. To test whether PPP holds, we
can test whether the real exchange rate is stationary.
\protect\hyperlink{ref-Kul}{Luintel} (\protect\hyperlink{ref-Kul}{2000})
argues that the power of unit-root tests is significantly higher when
using a panel data set as opposed to a univariate time series. The first
panel unit-root test that \protect\hyperlink{ref-Kul}{Luintel}
(\protect\hyperlink{ref-Kul}{2000}) runs is the Augmented Dickey-Fuller
test, which fails to reject the null hypothesis (there exists a unit
root, and therefore the process is nonstationary) for all the countries.
The only exception is for Pakistan in the post Bretton Woods period. My
replicated tests show similar results in \ref{ADF}, with most countries
failing to reject the null (with the exceptions occurring at a 5\% level
for four countries post Bretton Woods\footnote{India, Malaysia, Pakistan
  and Thailand}). Both \protect\hyperlink{ref-Kul}{Luintel}
(\protect\hyperlink{ref-Kul}{2000}) and my tests include a time trend,
but the non-stationary results hold when excluding this trend.

However, the Dickey-Fuller tests are known for their low power. Low
statistical power means we have a higher probability of committing a
type 2 error: failing to reject the null hypothesis when the alternative
hypothesis is true. In our context, this means we may be incorrectly
concluding that real exchange rates are nonstationary. Thus, we need to
consider other tests when testing for stationarity, which are discussed
below.

\begingroup\fontsize{10pt}{11pt}\selectfont
\begin{longtable}{lrrr}
\caption{Augmented Dickey-Fuller Tests} \\ 
  \toprule
Countries & Full Sample & Bretton Woods (1958:1-1973:3) & Post-Bretton Woods (1973:4-1989:6) \\ 
  \hline 
\endhead 
\hline 
{\footnotesize Continued on next page} 
\endfoot 
\endlastfoot 
 \midrule
India (Rupee) & -2.70 & -2.07 & -3.66 \\ 
  Sri Lanka (Rupee) & -3.22 & -2.11 & -2.44 \\ 
  Malaysia (Ringgit) & -1.47 & -2.12 & -3.77 \\ 
  Myanmar (Kyat) & -1.53 & -1.71 & -0.16 \\ 
  Pakistan (Rupee) & -3.35 & -2.63 & -5.91 \\ 
  Phillipines (Peso) & -3.09 & -2.07 & -3.20 \\ 
  Thailand (Baht) & -2.44 & -3.36 & -3.93 \\ 
   \bottomrule
\label{ADF}
\end{longtable}
\endgroup

As noted by \protect\hyperlink{ref-pes}{Breitung \& Pesaran}
(\protect\hyperlink{ref-pes}{2005: 18}), when using country data for
macroeconomic applications, there are often contemporaneous correlations
within the time series, which is a relevant concern for testing the PPP
hypothesis. There may be unobserved common factors or spatial spillover
effects, which need to be accounted for in the unit root test. Modelling
cross section dependence in panel data sets is still an emerging field,
but \protect\hyperlink{ref-im}{Pasaran, Im \& Shin}
(\protect\hyperlink{ref-im}{1997}) suggest that the appropriate test
statistic is the T-bar test based on cross-sectional demeaned
regressions. This is the approach that I take below (Im-Pesaran-Shin
T-bar test). I use the same lags as \protect\hyperlink{ref-Kul}{Luintel}
(\protect\hyperlink{ref-Kul}{2000}) for the first IPS T-bar test. For
the full sample: Malaysia(l) and Thailand(1), for the Bretton Woods
period: Thailand(1), and for the post Bretton Woods period: Malaysia(l)
and Thailand(1). The IPS test is run on the cross-sectionally demeaned
data.

The results (\ref{ips}) of the first IPS tests show that the null
hypothesis is rejected at a 1\% level of significance for the full
sample. This supports a stationary real exchange rate and therefore is
evidence towards the PPP. \protect\hyperlink{ref-Kul}{Luintel}
(\protect\hyperlink{ref-Kul}{2000: 173}) finds similar results. However,
for the Bretton Woods and Post Bretton Woods, my test fails to reject
the null hypothesis, whereas \protect\hyperlink{ref-Kul}{Luintel}
(\protect\hyperlink{ref-Kul}{2000}) rejects the null at 1\%. This
difference could be due to a difference in how the data was demeaned or
because I am testing a subset of currencies.

\begingroup\fontsize{12pt}{13pt}\selectfont
\begin{longtable}{llrrll}
\caption{IPS Panel Unit Root Tests (Tbar)} \\ 
  \toprule
Period & Test & T-statistic & P value & Trend & Lags \\ 
  \hline 
\endhead 
\hline 
{\footnotesize Continued on next page} 
\endfoot 
\endlastfoot 
 \midrule
Full Sample & IPS & -3.00 & 0.00 & No & Luintel \\ 
   & IPS & -2.14 & 0.02 & Yes & Luintel \\ 
  Bretton Woods & IPS & -0.76 & 0.22 & No & Luintel \\ 
   & IPS & -0.76 & 0.22 & Yes & Luintel \\ 
  Post Bretton Woods & IPS & -0.71 & 0.24 & No & Luintel \\ 
   & IPS & -0.71 & 0.24 & Yes & Luintel \\ 
   \bottomrule
\label{ips}
\end{longtable}
\endgroup

Next, I rerun the IPS test and use Akaike's information criterion (AIC)
to select the lags as a robustness check on the first IPS test. The
results (\ref{ipsAIC}) show that the null hypothesis is rejected at 1\%
for the full sample and for the post Bretton Woods period. This implies
there is no unit root present, and real exchange rates are
mean-reverting in the long run. This supports the findings of
\protect\hyperlink{ref-Kul}{Luintel}
(\protect\hyperlink{ref-Kul}{2000}). However, the test again fails to
reject the null for the Bretton Woods period, which is contrary to
\protect\hyperlink{ref-Kul}{Luintel}
(\protect\hyperlink{ref-Kul}{2000})'s results.

\begingroup\fontsize{12pt}{13pt}\selectfont
\begin{longtable}{llrrll}
\caption{IPS Panel Unit Root Tests (Tbar)} \\ 
  \toprule
Period & Test & T-statistic & P value & Trend & Lags \\ 
  \hline 
\endhead 
\hline 
{\footnotesize Continued on next page} 
\endfoot 
\endlastfoot 
 \midrule
Full Sample & IPS & -2.55 & 0.01 & No & AIC \\ 
   & IPS & 23.41 & 0.05 & Yes & AIC \\ 
  Bretton Woods & IPS & -0.33 & 0.37 & No & AIC \\ 
   & IPS & 11.87 & 0.62 & Yes & AIC \\ 
  Post Bretton Woods & IPS & -3.36 & 0.00 & No & AIC \\ 
   & IPS & 23.99 & 0.05 & Yes & AIC \\ 
   \bottomrule
\label{ipsAIC}
\end{longtable}
\endgroup

As a further robustness check on the results of
\protect\hyperlink{ref-Kul}{Luintel}
(\protect\hyperlink{ref-Kul}{2000}), I test the panel for unit roots
using the tests proposed by \protect\hyperlink{ref-lev}{Levin, Lin \&
James Chu} (\protect\hyperlink{ref-lev}{2002}) (LL),
\protect\hyperlink{ref-wu}{Maddala \& Wu}
(\protect\hyperlink{ref-wu}{1999}) (MadWu),
\protect\hyperlink{ref-had}{Hadri} (\protect\hyperlink{ref-had}{2002})
(Hadri). I used the package by \protect\hyperlink{ref-plm}{Millo}
(\protect\hyperlink{ref-plm}{2017}) to run these tests. For the
\protect\hyperlink{ref-lev}{Levin, Lin \& James Chu}
(\protect\hyperlink{ref-lev}{2002}) and
\protect\hyperlink{ref-wu}{Maddala \& Wu}
(\protect\hyperlink{ref-wu}{1999}) tests I again used AIC for the lag
selection. The \protect\hyperlink{ref-had}{Hadri}
(\protect\hyperlink{ref-had}{2002}) test directly tests for stationarity
and has as the null hypothesis that all the panels are (trend)
stationary.

The results presented in \ref{uunit} show that all
\protect\hyperlink{ref-lev}{Levin, Lin \& James Chu}
(\protect\hyperlink{ref-lev}{2002}) tests fail to reject the null
hypothesis (i.e.~there are unit roots present). The only
\protect\hyperlink{ref-wu}{Maddala \& Wu}
(\protect\hyperlink{ref-wu}{1999}) test that rejects the null hypothesis
is the one for post Bretton Woods when a trend is included. Otherwise
this test suggests that the real exchange rate is nonstationary. All
\protect\hyperlink{ref-had}{Hadri} (\protect\hyperlink{ref-had}{2002})
tests reject the null hypothesis at a 1\% level of significance. A
rejection of the null here is an indication of nonstationarity. These
results can be interpereted in two ways. They can suggest that real
exchange rates are nonstationary, the PPP doesn't hold empirically and
\protect\hyperlink{ref-Kul}{Luintel}
(\protect\hyperlink{ref-Kul}{2000})'s results are not robust. Or these
results support \protect\hyperlink{ref-Kul}{Luintel}
(\protect\hyperlink{ref-Kul}{2000})'s claim that the IPS test is the
correct test to use because other unit root tests are too weak to
correctly identify stationary processes.

\begingroup\fontsize{12pt}{13pt}\selectfont
\begin{longtable}{llrrll}
\caption{Various Panel Unit Root Tests} \\ 
  \toprule
Period & Test & T-statistic & P value & Trend & Lags \\ 
  \hline 
\endhead 
\hline 
{\footnotesize Continued on next page} 
\endfoot 
\endlastfoot 
 \midrule
Full Sample & LL & -0.51 & 0.31 & No & AIC \\ 
   & LL & 1.01 & 0.84 & Yes & AIC \\ 
   & MadWu & 17.47 & 0.23 & No & AIC \\ 
   & MadWu & 13.90 & 0.46 & Yes & AIC \\ 
   & Hadri & 233.20 & 0.00 & No & NA \\ 
   & Hadri & 171.05 & 0.00 & Yes & NA \\ 
  Bretton Woods & LL & -0.32 & 0.37 & No & AIC \\ 
   & LL & 1.44 & 0.92 & Yes & AIC \\ 
   & MadWu & 16.80 & 0.27 & No & AIC \\ 
   & MadWu & 8.94 & 0.84 & Yes & AIC \\ 
   & Hadri & 165.47 & 0.00 & No & NA \\ 
   & Hadri & 100.18 & 0.00 & Yes & NA \\ 
  Post Bretton Woods & LL & 0.47 & 0.68 & No & AIC \\ 
   & LL & -0.94 & 0.17 & Yes & AIC \\ 
   & MadWu & 10.51 & 0.72 & No & AIC \\ 
   & MadWu & 51.41 & 0.00 & Yes & AIC \\ 
   & Hadri & 224.04 & 0.00 & No & NA \\ 
   & Hadri & 108.94 & 0.00 & Yes & NA \\ 
   \bottomrule
\label{uunit}
\end{longtable}
\endgroup

As a final check on the stationarity of real exchange rates, I employ a
panel bootstrap group-mean union test as proposed by
\protect\hyperlink{ref-pal}{Palm, Smeekes \& Urbain}
(\protect\hyperlink{ref-pal}{2011}). The test has a null hypothesis that
all series have a unit root. If the null is rejected then some
proportion of the series is stationary. I also ran the test on the
cross-section demeaned data (in addition to the non-demeaned data) for
the full sample. The results are shown in \ref{Boot}. The test fails to
reject the null for both series, which suggests the panel is
non-stationary. This undermines the results of
\protect\hyperlink{ref-Kul}{Luintel}
(\protect\hyperlink{ref-Kul}{2000}). However, the number of runs for the
test was 1000\footnote{My laptop struggled with runs higher than this
  unfortunately.}, which is quite low for bootstrapping and the test may
be giving inaccurate results.

\begin{table}[H]
\centering
\caption{Bootstrapped panel unit root tests} 
\label{Boot}
\begin{tabular}{lrr}
  \hline
Test & Test Statistic & P value \\ 
  \hline
Panel & -0.82 & 0.28 \\ 
  Panel Demeaned & -0.86 & 0.19 \\ 
   \hline
\end{tabular}
\end{table}

\hypertarget{variance-ratio-test}{%
\section{\texorpdfstring{Variance Ratio Test
\label{Var}}{Variance Ratio Test }}\label{variance-ratio-test}}

\protect\hyperlink{ref-Kul}{Luintel} (\protect\hyperlink{ref-Kul}{2000:
174}) makes use of the variance ratio test to examine the persistence in
real exchange rates. The variance ratio \(V^k\) is defined as:

\[
V^k = \frac{Var(y_t - y_{t-k})}{k \times Var(y_t - y_{t-1}) } 
\]

where k is the lag length, \(Var(y_t - y_{t-k})\) and
\(Var(y_t - y_{t-1})\) are the variances of kth difference and the first
difference of a time series \(y_t\).
\protect\hyperlink{ref-Kul}{Luintel} (\protect\hyperlink{ref-Kul}{2000:
174}) goes into detail of the intuition and the interpretation of this
test. I replicate this test, and table (\ref{v}) shows results for the
full sample for up to 20 months. The results of the variance ratio test
for the Bretton Woods period and post Bretton Woods period (for up to 20
months\footnote{The results for 190 months are available upon request;
  it has been omitted to save space}) can be found in the Appendix
(\ref{A}). I find the same variance ratios as
\protect\hyperlink{ref-Kul}{Luintel} (\protect\hyperlink{ref-Kul}{2000})
for all the countries. My standard errors (se) were slightly larger
because the formula I used for the variance was different\footnote{I
  tried to replicate the variance formula as it is in the paper but it
  kept breaking the code of the function I built, unfortunately} but the
qualitative results remain similar. For all the countries except
Myanmar, the variance ratios are all significantly different from zero
for k ranging from 1 - 70, and the cut-off lag length is 70 for all the
countries. According to \protect\hyperlink{ref-Kul}{Luintel}
(\protect\hyperlink{ref-Kul}{2000}), this suggests that the real
exchange rates (for the countries other than Myanmar) are stationary.
The two sub-periods show similar results (\ref{A}).

\begingroup\fontsize{12pt}{13pt}\selectfont
\begin{longtable}{lrrrrrrr}
\caption{Variance Ratio Test for Full Sample Up to month 20} \\ 
  \toprule
Months & India & SriLanka & Malaysia & Myanmar & Pakistan & Philippines & Thailand \\ 
  \hline 
\endhead 
\hline 
{\footnotesize Continued on next page} 
\endfoot 
\endlastfoot 
 \midrule
1 & 1.00 & 1.00 & 1.00 & 1.00 & 1.00 & 1.00 & 1.00 \\ 
  se & 0.10 & 0.10 & 0.10 & 0.10 & 0.10 & 0.10 & 0.10 \\ 
  2 & 1.00 & 0.95 & 0.79 & 1.04 & 0.91 & 0.91 & 0.74 \\ 
  se & 0.10 & 0.10 & 0.10 & 0.10 & 0.10 & 0.10 & 0.10 \\ 
  3 & 1.02 & 0.86 & 0.79 & 1.05 & 0.81 & 0.86 & 0.68 \\ 
  se & 0.10 & 0.10 & 0.10 & 0.10 & 0.10 & 0.10 & 0.10 \\ 
  4 & 1.01 & 0.87 & 0.75 & 1.00 & 0.71 & 0.82 & 0.58 \\ 
  se & 0.10 & 0.10 & 0.10 & 0.10 & 0.10 & 0.10 & 0.10 \\ 
  5 & 0.95 & 0.89 & 0.73 & 0.98 & 0.65 & 0.80 & 0.52 \\ 
  se & 0.10 & 0.10 & 0.10 & 0.10 & 0.10 & 0.10 & 0.10 \\ 
  6 & 0.91 & 0.90 & 0.73 & 0.95 & 0.61 & 0.77 & 0.48 \\ 
  se & 0.10 & 0.10 & 0.10 & 0.10 & 0.10 & 0.10 & 0.10 \\ 
  7 & 0.86 & 0.91 & 0.69 & 0.93 & 0.58 & 0.76 & 0.44 \\ 
  se & 0.10 & 0.10 & 0.10 & 0.10 & 0.10 & 0.10 & 0.10 \\ 
  8 & 0.83 & 0.90 & 0.69 & 0.92 & 0.56 & 0.77 & 0.42 \\ 
  se & 0.10 & 0.10 & 0.10 & 0.10 & 0.10 & 0.10 & 0.10 \\ 
  9 & 0.81 & 0.89 & 0.66 & 0.90 & 0.53 & 0.81 & 0.40 \\ 
  se & 0.10 & 0.10 & 0.10 & 0.10 & 0.10 & 0.10 & 0.10 \\ 
  10 & 0.81 & 0.88 & 0.63 & 0.89 & 0.50 & 0.79 & 0.39 \\ 
  se & 0.10 & 0.10 & 0.10 & 0.10 & 0.10 & 0.10 & 0.10 \\ 
  11 & 0.81 & 0.88 & 0.61 & 0.91 & 0.49 & 0.79 & 0.37 \\ 
  se & 0.10 & 0.10 & 0.10 & 0.10 & 0.10 & 0.10 & 0.10 \\ 
  12 & 0.83 & 0.88 & 0.57 & 0.95 & 0.46 & 0.78 & 0.37 \\ 
  se & 0.10 & 0.10 & 0.10 & 0.10 & 0.10 & 0.10 & 0.10 \\ 
  13 & 0.86 & 0.87 & 0.57 & 0.96 & 0.47 & 0.79 & 0.37 \\ 
  se & 0.10 & 0.10 & 0.10 & 0.10 & 0.10 & 0.10 & 0.10 \\ 
  14 & 0.88 & 0.87 & 0.57 & 0.98 & 0.48 & 0.79 & 0.37 \\ 
  se & 0.10 & 0.10 & 0.10 & 0.10 & 0.10 & 0.10 & 0.10 \\ 
  15 & 0.90 & 0.88 & 0.57 & 1.00 & 0.48 & 0.80 & 0.37 \\ 
  se & 0.10 & 0.10 & 0.10 & 0.10 & 0.10 & 0.10 & 0.10 \\ 
  16 & 0.92 & 0.88 & 0.57 & 1.02 & 0.49 & 0.79 & 0.37 \\ 
  se & 0.10 & 0.10 & 0.10 & 0.10 & 0.10 & 0.10 & 0.10 \\ 
  17 & 0.92 & 0.89 & 0.58 & 1.03 & 0.49 & 0.78 & 0.36 \\ 
  se & 0.10 & 0.10 & 0.10 & 0.10 & 0.10 & 0.10 & 0.10 \\ 
  18 & 0.92 & 0.88 & 0.59 & 1.04 & 0.49 & 0.77 & 0.36 \\ 
  se & 0.10 & 0.10 & 0.10 & 0.10 & 0.10 & 0.10 & 0.10 \\ 
  19 & 0.91 & 0.88 & 0.60 & 1.05 & 0.50 & 0.76 & 0.36 \\ 
  se & 0.10 & 0.10 & 0.10 & 0.10 & 0.10 & 0.10 & 0.10 \\ 
  20 & 0.90 & 0.87 & 0.62 & 1.08 & 0.52 & 0.75 & 0.36 \\ 
  se & 0.10 & 0.10 & 0.10 & 0.10 & 0.10 & 0.10 & 0.10 \\ 
   \bottomrule
\label{v}
\end{longtable}
\endgroup

\hypertarget{conclusion}{%
\section{Conclusion}\label{conclusion}}

\protect\hyperlink{ref-Kul}{Luintel} (\protect\hyperlink{ref-Kul}{2000})
performs a number of tests to show that real exchange rates are
stationary and thus PPP holds empirically. My replication of the IPS
test and variance ratio test support the results of the paper. However,
the robustness checks of using other panel unit root tests show that
real exchange rates are nonstationary. This undermines the results and
conclusions reached by \protect\hyperlink{ref-Kul}{Luintel}
(\protect\hyperlink{ref-Kul}{2000}). An interesting study would be to
extend the panel (in terms of the time dimension and include more
countries) and apply unit root testing to ascertain whether real
exchange rates are stationary.

\newpage

\hypertarget{references}{%
\section*{References}\label{references}}
\addcontentsline{toc}{section}{References}

\hypertarget{refs}{}
\begin{CSLReferences}{1}{0}
\leavevmode\hypertarget{ref-pes}{}%
Breitung, J. \& Pesaran, M.H. 2005. \emph{Unit roots and cointegration
in panels}. Deutsche Bundesbank. {[}Online{]}, Available:
\url{https://ideas.repec.org/p/zbw/bubdp1/4236.html}.

\leavevmode\hypertarget{ref-puz}{}%
El-Gamal, M.A. \& Ryu, D. 2006. Short-memory and the PPP hypothesis.
\emph{Journal of Economic Dynamics and Control}. 30(3):361--391.
{[}Online{]}, Available:
\url{http://www.sciencedirect.com/science/article/B6V85-4HD8B3Y-1/1/b7865989592c6f1ee4b79a80036183c9}.

\leavevmode\hypertarget{ref-had}{}%
Hadri, K. 2002. Testing for stationarity in heterogeneous panel data.
\emph{The Econometrics Journal}. 3(2):148--161.

\leavevmode\hypertarget{ref-Texevier}{}%
Katzke, N.F. 2017. \emph{{Texevier}: {P}ackage to create elsevier
templates for rmarkdown}. Stellenbosch, South Africa: Bureau for
Economic Research.

\leavevmode\hypertarget{ref-lev}{}%
Levin, A., Lin, C.-F. \& James Chu, C.-S. 2002. Unit root tests in panel
data: Asymptotic and finite-sample properties. \emph{Journal of
Econometrics}. 108(1):1--24. {[}Online{]}, Available:
\url{https://EconPapers.repec.org/RePEc:eee:econom:v:108:y:2002:i:1:p:1-24}.

\leavevmode\hypertarget{ref-Kul}{}%
Luintel, K.B. 2000. Real exchange rate behaviour: Evidence from black
markets. \emph{Journal of Applied Econometrics}. 15(2):161--185.
{[}Online{]}, Available: \url{http://www.jstor.org/stable/2678529}.

\leavevmode\hypertarget{ref-wu}{}%
Maddala, G.S. \& Wu, S. 1999. A comparative study of unit root tests
with panel data and a new simple test. \emph{Oxford Bulletin of
Economics and Statistics}. 61(S1):631--652. {[}Online{]}, Available:
\url{https://EconPapers.repec.org/RePEc:bla:obuest:v:61:y:1999:i:s1:p:631-652}.

\leavevmode\hypertarget{ref-plm}{}%
Millo, G. 2017. Robust standard error estimators for panel models: A
unifying approach. \emph{Journal of Statistical Software}. 82(3):1--27.

\leavevmode\hypertarget{ref-pal}{}%
Palm, F., Smeekes, S. \& Urbain, J.-P. 2011. Cross-sectional dependence
robust block bootstrap panel unit root tests. \emph{Journal of
Econometrics}. 163(1):85--104. {[}Online{]}, Available:
\url{https://EconPapers.repec.org/RePEc:eee:econom:v:163:y:2011:i:1:p:85-104}.

\leavevmode\hypertarget{ref-im}{}%
Pasaran, M.H., Im, K.S. \& Shin, Y. 1997. \emph{Testing for unit roots
in heterogeneous panels}. Faculty of Economics, University of Cambridge.
{[}Online{]}, Available:
\url{https://EconPapers.repec.org/RePEc:cam:camdae:9526}.

\leavevmode\hypertarget{ref-rog}{}%
Rogoff, K. 1996. The purchasing power parity puzzle. \emph{Journal of
Economic Literature}. 34:647--68.

\end{CSLReferences}

\newpage

\hypertarget{appendix}{%
\section*{\texorpdfstring{Appendix
\label{A}}{Appendix }}\label{appendix}}
\addcontentsline{toc}{section}{Appendix \label{A}}

\begingroup\fontsize{12pt}{13pt}\selectfont
\begin{longtable}{lrrrrrrr}
\caption{Variance Ratio Test for Bretton Woods period up to month 20} \\ 
  \toprule
Months & India & SriLanka & Malaysia & Myanmar & Pakistan & Philippines & Thailand \\ 
  \hline 
\endhead 
\hline 
{\footnotesize Continued on next page} 
\endfoot 
\endlastfoot 
 \midrule
1 & 1.00 & 1.00 & 1.00 & 1.00 & 1.00 & 1.00 & 1.00 \\ 
  se & 0.15 & 0.15 & 0.15 & 0.15 & 0.15 & 0.15 & 0.15 \\ 
  2 & 1.06 & 0.88 & 0.80 & 1.03 & 1.01 & 1.02 & 0.79 \\ 
  se & 0.15 & 0.15 & 0.15 & 0.15 & 0.15 & 0.15 & 0.15 \\ 
  3 & 1.03 & 0.80 & 0.73 & 1.01 & 0.92 & 0.90 & 0.72 \\ 
  se & 0.15 & 0.15 & 0.15 & 0.15 & 0.15 & 0.15 & 0.15 \\ 
  4 & 0.99 & 0.77 & 0.66 & 0.95 & 0.76 & 0.84 & 0.61 \\ 
  se & 0.15 & 0.15 & 0.15 & 0.15 & 0.15 & 0.15 & 0.15 \\ 
  5 & 0.92 & 0.79 & 0.59 & 0.93 & 0.61 & 0.81 & 0.50 \\ 
  se & 0.15 & 0.15 & 0.15 & 0.15 & 0.15 & 0.15 & 0.15 \\ 
  6 & 0.88 & 0.80 & 0.56 & 0.91 & 0.55 & 0.79 & 0.47 \\ 
  se & 0.15 & 0.15 & 0.15 & 0.15 & 0.15 & 0.15 & 0.15 \\ 
  7 & 0.84 & 0.80 & 0.53 & 0.90 & 0.50 & 0.79 & 0.39 \\ 
  se & 0.15 & 0.15 & 0.15 & 0.15 & 0.15 & 0.15 & 0.15 \\ 
  8 & 0.82 & 0.80 & 0.55 & 0.89 & 0.49 & 0.81 & 0.36 \\ 
  se & 0.15 & 0.15 & 0.15 & 0.15 & 0.15 & 0.15 & 0.15 \\ 
  9 & 0.80 & 0.80 & 0.55 & 0.88 & 0.44 & 0.83 & 0.36 \\ 
  se & 0.15 & 0.15 & 0.15 & 0.15 & 0.15 & 0.15 & 0.15 \\ 
  10 & 0.80 & 0.78 & 0.56 & 0.87 & 0.39 & 0.82 & 0.36 \\ 
  se & 0.15 & 0.15 & 0.15 & 0.15 & 0.15 & 0.15 & 0.15 \\ 
  11 & 0.79 & 0.78 & 0.56 & 0.90 & 0.36 & 0.81 & 0.37 \\ 
  se & 0.15 & 0.15 & 0.15 & 0.15 & 0.15 & 0.15 & 0.15 \\ 
  12 & 0.80 & 0.78 & 0.53 & 0.96 & 0.34 & 0.82 & 0.35 \\ 
  se & 0.15 & 0.15 & 0.15 & 0.15 & 0.15 & 0.15 & 0.15 \\ 
  13 & 0.83 & 0.76 & 0.53 & 0.98 & 0.35 & 0.84 & 0.35 \\ 
  se & 0.15 & 0.15 & 0.15 & 0.15 & 0.15 & 0.15 & 0.15 \\ 
  14 & 0.86 & 0.74 & 0.55 & 1.00 & 0.36 & 0.85 & 0.34 \\ 
  se & 0.15 & 0.15 & 0.15 & 0.15 & 0.15 & 0.15 & 0.15 \\ 
  15 & 0.90 & 0.74 & 0.56 & 1.04 & 0.35 & 0.87 & 0.32 \\ 
  se & 0.15 & 0.15 & 0.15 & 0.15 & 0.15 & 0.15 & 0.15 \\ 
  16 & 0.88 & 0.72 & 0.56 & 1.07 & 0.34 & 0.87 & 0.31 \\ 
  se & 0.15 & 0.15 & 0.15 & 0.15 & 0.15 & 0.15 & 0.15 \\ 
  17 & 0.89 & 0.71 & 0.56 & 1.09 & 0.33 & 0.87 & 0.30 \\ 
  se & 0.15 & 0.15 & 0.15 & 0.15 & 0.15 & 0.15 & 0.15 \\ 
  18 & 0.89 & 0.71 & 0.56 & 1.10 & 0.34 & 0.88 & 0.31 \\ 
  se & 0.15 & 0.15 & 0.15 & 0.15 & 0.15 & 0.15 & 0.15 \\ 
  19 & 0.87 & 0.70 & 0.57 & 1.11 & 0.35 & 0.88 & 0.31 \\ 
  se & 0.15 & 0.15 & 0.15 & 0.15 & 0.15 & 0.15 & 0.15 \\ 
  20 & 0.84 & 0.69 & 0.58 & 1.15 & 0.36 & 0.89 & 0.32 \\ 
  se & 0.15 & 0.15 & 0.15 & 0.15 & 0.15 & 0.15 & 0.15 \\ 
   \bottomrule
\end{longtable}
\endgroup

\begingroup\fontsize{12pt}{13pt}\selectfont
\begin{longtable}{lrrrrrrr}
\caption{Variance Ratio Test for post Bretton Woods period up to 20 months} \\ 
  \toprule
Months & India & SriLanka & Malaysia & Myanmar & Pakistan & Philippines & Thailand \\ 
  \hline 
\endhead 
\hline 
{\footnotesize Continued on next page} 
\endfoot 
\endlastfoot 
 \midrule
1 & 1.00 & 1.00 & 1.00 & 1.00 & 1.00 & 1.00 & 1.00 \\ 
  se & 0.14 & 0.14 & 0.14 & 0.14 & 0.14 & 0.14 & 0.14 \\ 
  2 & 0.95 & 1.01 & 0.78 & 1.05 & 0.78 & 0.85 & 0.71 \\ 
  se & 0.14 & 0.14 & 0.14 & 0.14 & 0.14 & 0.14 & 0.14 \\ 
  3 & 0.99 & 0.91 & 0.80 & 1.14 & 0.68 & 0.84 & 0.66 \\ 
  se & 0.14 & 0.14 & 0.14 & 0.14 & 0.14 & 0.14 & 0.14 \\ 
  4 & 0.98 & 0.93 & 0.75 & 1.14 & 0.61 & 0.82 & 0.58 \\ 
  se & 0.14 & 0.14 & 0.14 & 0.14 & 0.14 & 0.14 & 0.14 \\ 
  5 & 0.93 & 0.94 & 0.76 & 1.12 & 0.60 & 0.81 & 0.53 \\ 
  se & 0.14 & 0.14 & 0.14 & 0.14 & 0.14 & 0.14 & 0.14 \\ 
  6 & 0.88 & 0.94 & 0.73 & 1.06 & 0.54 & 0.78 & 0.49 \\ 
  se & 0.14 & 0.14 & 0.14 & 0.14 & 0.14 & 0.14 & 0.14 \\ 
  7 & 0.82 & 0.94 & 0.69 & 1.02 & 0.50 & 0.76 & 0.45 \\ 
  se & 0.14 & 0.14 & 0.14 & 0.14 & 0.14 & 0.14 & 0.14 \\ 
  8 & 0.77 & 0.92 & 0.68 & 1.00 & 0.45 & 0.76 & 0.44 \\ 
  se & 0.14 & 0.14 & 0.14 & 0.14 & 0.14 & 0.14 & 0.14 \\ 
  9 & 0.75 & 0.89 & 0.62 & 0.98 & 0.40 & 0.81 & 0.40 \\ 
  se & 0.15 & 0.15 & 0.15 & 0.15 & 0.15 & 0.15 & 0.15 \\ 
  10 & 0.75 & 0.86 & 0.60 & 0.98 & 0.39 & 0.79 & 0.38 \\ 
  se & 0.15 & 0.15 & 0.15 & 0.15 & 0.15 & 0.15 & 0.15 \\ 
  11 & 0.75 & 0.85 & 0.55 & 0.98 & 0.38 & 0.79 & 0.34 \\ 
  se & 0.15 & 0.15 & 0.15 & 0.15 & 0.15 & 0.15 & 0.15 \\ 
  12 & 0.76 & 0.84 & 0.51 & 0.99 & 0.37 & 0.78 & 0.33 \\ 
  se & 0.15 & 0.15 & 0.15 & 0.15 & 0.15 & 0.15 & 0.15 \\ 
  13 & 0.76 & 0.82 & 0.47 & 1.00 & 0.39 & 0.78 & 0.32 \\ 
  se & 0.15 & 0.15 & 0.15 & 0.15 & 0.15 & 0.15 & 0.15 \\ 
  14 & 0.75 & 0.83 & 0.45 & 0.99 & 0.40 & 0.77 & 0.32 \\ 
  se & 0.15 & 0.15 & 0.15 & 0.15 & 0.15 & 0.15 & 0.15 \\ 
  15 & 0.73 & 0.83 & 0.43 & 0.98 & 0.39 & 0.77 & 0.31 \\ 
  se & 0.15 & 0.15 & 0.15 & 0.15 & 0.15 & 0.15 & 0.15 \\ 
  16 & 0.72 & 0.83 & 0.42 & 0.99 & 0.39 & 0.75 & 0.31 \\ 
  se & 0.15 & 0.15 & 0.15 & 0.15 & 0.15 & 0.15 & 0.15 \\ 
  17 & 0.72 & 0.83 & 0.42 & 0.98 & 0.38 & 0.74 & 0.30 \\ 
  se & 0.15 & 0.15 & 0.15 & 0.15 & 0.15 & 0.15 & 0.15 \\ 
  18 & 0.71 & 0.81 & 0.42 & 0.98 & 0.38 & 0.72 & 0.28 \\ 
  se & 0.15 & 0.15 & 0.15 & 0.15 & 0.15 & 0.15 & 0.15 \\ 
  19 & 0.68 & 0.80 & 0.42 & 0.99 & 0.38 & 0.70 & 0.27 \\ 
  se & 0.15 & 0.15 & 0.15 & 0.15 & 0.15 & 0.15 & 0.15 \\ 
  20 & 0.64 & 0.78 & 0.41 & 0.99 & 0.39 & 0.68 & 0.25 \\ 
  se & 0.15 & 0.15 & 0.15 & 0.15 & 0.15 & 0.15 & 0.15 \\ 
   \bottomrule
\end{longtable}
\endgroup

\bibliography{Tex/ref}





\end{document}
